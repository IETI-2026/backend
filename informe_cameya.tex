\documentclass[12pt,a4paper,spanish]{article}
\usepackage[T1]{fontenc}
\usepackage[utf8]{inputenc}
\usepackage[spanish]{babel}
\usepackage[margin=2.5cm]{geometry}
\usepackage{graphicx}
\usepackage{hyperref}
\usepackage{xcolor}
\usepackage{listings}
\usepackage{float}
\usepackage{fancyhdr}
\usepackage{tikz}
\usepackage{array}
\usepackage{booktabs}

% Configuración de listings para código
\lstset{
    basicstyle=\ttfamily\small,
    breaklines=true,
    postbreak=\mbox{\textcolor{red}{$\hookrightarrow$}\space},
    backgroundcolor=\color{gray!10},
    frame=single,
    rulecolor=\color{gray!20}
}

% Configuración de encabezados y pies
\pagestyle{fancy}
\fancyhf{}
\rhead{\textit{CameYa / CameYo}}
\lhead{\textit{Plataforma de Servicios Bajo Demanda}}
\cfoot{\thepage}

% Colores personalizados
\definecolor{darkblue}{rgb}{0.0, 0.0, 0.5}
\definecolor{accentblue}{rgb}{0.2, 0.4, 0.8}

% Comando para títulos de sección personalizado
\newcommand{\sectionline}{%
    \noindent\rule[0.5ex]{\linewidth}{1pt}
}

\title{
    \textbf{\Huge CameYa / CameYo}\\
    \vspace{0.5cm}
    \textbf{\Large Plataforma Inteligente de Servicios Bajo Demanda}\\
    \vspace{1cm}
    \textit{\large Especificación Base, Arquitectura y Propuesta de Implementación}
}

\author{
    Proyecto Universitario\\
    \textit{Ingeniería de Software Avanzada}
}

\date{\today}

\begin{document}

% ==================== PORTADA ====================
\maketitle
\vspace{2cm}

\begin{abstract}
    \noindent\textbf{\large Resumen Ejecutivo}
    \vspace{0.5cm}
    
    CameYa/CameYo es una plataforma digital de intermediación de servicios domésticos y técnicos bajo demanda, diseñada para conectar usuarios con trabajadores independientes verificados mediante algoritmos de asignación inteligente e inteligencia artificial. Este documento presenta la especificación completa del proyecto, incluyendo problemática identificada, objetivos, requisitos funcionales, arquitectura tecnológica recomendada y propuesta de implementación a nivel empresarial y académico.
    
    \vspace{0.5cm}
    \noindent\textbf{Palabras clave:} Plataforma digital, servicios bajo demanda, inteligencia artificial, microservicios, arquitectura de software, NLP, geolocalización.
\end{abstract}

\newpage
\tableofcontents
\newpage

% ==================== 1. INTRODUCCIÓN ====================
\section{Introducción}

\subsection{Contexto}

El mercado de servicios domésticos y técnicos representa una oportunidad significativa de transformación digital en contextos locales (especialmente en países como Colombia). Actualmente, existe un vacío entre la demanda de servicios confiables y la capacidad de los trabajadores independientes para formalizarse y acceder a un mercado estructurado.

Plataformas como Uber, Rappi y Workana han demostrado que la intermediación digital bajo demanda es viable y escalable. Sin embargo, no existe en el contexto local una solución integral especializada en servicios técnicos del hogar que combine:

\begin{itemize}
    \item Solicitud en lenguaje natural
    \item Clasificación automática mediante IA
    \item Asignación óptima
    \item Verificación de antecedentes
    \item Pagos digitales integrados
\end{itemize}

\subsection{Motivación del proyecto}

Este proyecto nace de la necesidad académica y empresarial de:

\begin{enumerate}
    \item Demostrar aplicación real de técnicas de ingeniería de software avanzada
    \item Implementar arquitecturas modernas y escalables
    \item Integrar inteligencia artificial en soluciones prácticas
    \item Crear una base de startup viable
\end{enumerate}

\newpage

% ==================== 2. PROBLEMÁTICA ====================
\section{Problemática Identificada}

\subsection{Problema Central}

Existe una \textbf{desconexión estructural} entre la demanda de servicios domésticos/técnicos y la oferta confiable de estos servicios, generando:

\begin{center}
\begin{tabular}{|c|c|}
    \toprule
    \textbf{Efecto} & \textbf{Impacto} \\
    \midrule
    Desconfianza & Usuarios evitan contratar servicios informales \\
    Ineficiencia & Tiempos largos de búsqueda y asignación \\
    Informalidad & Trabajadores sin visibilidad formal \\
    Pérdida económica & Estafas, sobrecostos, incumplimientos \\
    \bottomrule
\end{tabular}
\end{center}

\subsection{Problemas Específicos por Actor}

\subsubsection{Usuarios (Clientes)}

\begin{itemize}
    \item No existe canal confiable para contratar servicios
    \item Precios poco claros o negociados informalmente
    \item Incertidumbre sobre: quién llega, cuándo, qué tan confiable es
    \item Riesgos de seguridad sin trazabilidad
    \item Falta de recibos o comprobantes de pago
\end{itemize}

\subsubsection{Trabajadores Independientes}

\begin{itemize}
    \item Ausencia de vitrina digital
    \item Dependencia del voz a voz (sin escalabilidad)
    \item Ingresos variables e impredecibles
    \item Imposibilidad de construir reputación formal verificable
    \item Dificultades para recibir pagos digitales
    \item Vulnerabilidad ante falta de contratos formales
\end{itemize}

\subsubsection{Administrador del Sistema}

\begin{itemize}
    \item Necesidad de automatizar procesos de clasificación
    \item Complejidad de asignación óptima manualmente
    \item Falta de herramientas de moderación
    \item Dificultad en verificación de antecedentes
\end{itemize}

\newpage

% ==================== 3. OBJETIVOS ====================
\section{Objetivos del Proyecto}

\subsection{Objetivo General}

\begin{quotation}
    \noindent Diseñar y desarrollar una plataforma tecnológica escalable que conecte usuarios con trabajadores independientes de manera \textbf{segura, eficiente e inteligente}, mediante el uso de \textbf{inteligencia artificial, geolocalización y pagos digitales}.
\end{quotation}

\subsection{Objetivos Específicos}

\begin{enumerate}
    \item Facilitar la solicitud de servicios mediante \textbf{procesamiento de lenguaje natural}
    
    \item Automatizar la \textbf{clasificación y priorización} de solicitudes con precisión $\geq 90\%$
    
    \item Optimizar la \textbf{asignación de prestadores} mediante algoritmo de scoring multicriterio
    
    \item Garantizar \textbf{trazabilidad completa} del servicio (estados, timestamps, GPS)
    
    \item Formalizar la \textbf{reputación del trabajador} mediante historial verificable
    
    \item Integrar \textbf{medios de pago digitales locales} (Nequi, Daviplata, Bre-B)
    
    \item Reducir tiempos de respuesta a \textbf{menos de 5 minutos} en zonas de alta cobertura
    
    \item Implementar \textbf{verificación de antecedentes} legal y segura
\end{enumerate}

\newpage

% ==================== 4. ALCANCE ====================
\section{Alcance del Sistema}

\subsection{Incluye (MVP y Fase 1)}

\begin{itemize}
    \item Aplicación móvil nativa (Flutter) para usuarios y prestadores
    \item Backend centralizado con arquitectura de microservicios
    \item Motor de IA para clasificación y asignación
    \item Sistema de pagos integrado
    \item Sistema de evaluación y reputación
    \item Panel administrativo web (React)
    \item Tracking GPS en tiempo real
    \item Notificaciones push
    \item Dashboard analítico básico
\end{itemize}

\subsection{No Incluye (Fases Posteriores)}

\begin{itemize}
    \item Contratación empresarial a gran escala
    \item Suscripciones avanzadas (Fase 2)
    \item Facturación electrónica automática (integraciones futuras)
    \item IA predictiva compleja de demanda
    \item Expansion a otros países (Fase 3+)
\end{itemize}

\newpage

% ==================== 5. ACTORES Y ROLES ====================
\section{Actores del Sistema}

\subsection{Usuario Solicitante}

\begin{itemize}
    \item \textbf{Tipo:} Persona natural
    \item \textbf{Contextos:} Hogar, oficina pequeña, comercio local
    \item \textbf{Intereses:} Confianza, rapidez, precio justo, transparencia
\end{itemize}

\subsection{Trabajador Independiente (Prestador)}

\begin{itemize}
    \item \textbf{Tipo:} Persona natural o microempresa
    \item \textbf{Profesiones:} Técnico, plomero, electricista, carpintero, limpieza, servicios generales
    \item \textbf{Intereses:} Visibilidad, ingresos estables, construcción de reputación, pagos seguros
\end{itemize}

\subsection{Administrador del Sistema}

\begin{itemize}
    \item \textbf{Función:} Operador de la plataforma
    \item \textbf{Responsabilidades:} 
        \begin{itemize}
            \item Moderación
            \item Verificación de prestadores
            \item Soporte
            \item Análisis y reportes
            \item Resolución de conflictos
        \end{itemize}
\end{itemize}

\newpage

% ==================== 6. FUNCIONALIDADES ====================
\section{Funcionalidades Detalladas}

\subsection{Registro y Autenticación}

\subsubsection{Usuario Solicitante}

Registro con datos mínimos:
\begin{itemize}
    \item Nombre completo
    \item Número de documento
    \item Teléfono móvil
    \item Correo electrónico
    \item Dirección
\end{itemize}

Autenticación mediante:
\begin{itemize}
    \item Email + contraseña
    \item SMS OTP
    \item OAuth (Google, Facebook)
\end{itemize}

\subsubsection{Trabajador Independiente}

Registro extendido con:
\begin{itemize}
    \item Documento (COI, Cédula)
    \item Foto verificable
    \item Servicios ofrecidos (taxonomía)
    \item Zona de trabajo (radio de cobertura)
    \item Certificado judicial (documento)
    \item Cuenta bancaria o digital para pagos
\end{itemize}

Estados de verificación:
\begin{itemize}
    \item \textbf{Pendiente:} Registro completado, en revisión
    \item \textbf{Aprobado:} Verificación exitosa, apto para recibir solicitudes
    \item \textbf{Suspendido:} Violaciones o historias negativas
    \item \textbf{Rechazado:} No cumple criterios
\end{itemize}

\subsection{Solicitud de Servicio (Núcleo del Sistema)}

\subsubsection{Flujo de Solicitud}

\begin{enumerate}
    \item \textbf{Descripción en lenguaje natural:} El usuario describe el problema libremente
    
    \textit{Ejemplo:} ``Tengo una fuga de agua en el baño y es urgente''
    
    \item \textbf{Ubicación:} Sistema captura coordenadas GPS
    
    \item \textbf{Urgencia (opcional):} Usuario indica si es urgente
    
    \item \textbf{Rango de presupuesto (opcional):} Usuario propone rango
\end{enumerate}

\subsubsection{Procesamiento por IA}

El motor de IA (servicio separado en Python + FastAPI) realiza:

\begin{lstlisting}[language=json]
Input: "Tengo una fuga de agua en el baño y es urgente"

Output: {
  "categoria": "plomeria",
  "subcategoria": "reparacion_urgente",
  "urgencia": "alta",
  "prioridad": 0.92,
  "palabras_clave": ["agua", "fuga", "baño"],
  "estimado_tiempo": 45,
  "estimado_costo": 80000
}
\end{lstlisting}

Etapas de procesamiento:
\begin{itemize}
    \item Limpieza y normalización de texto
    \item Clasificación de servicio (ML supervisado)
    \item Detección de urgencia (reglas + ML)
    \item Extracción de palabras clave (NLP)
    \item Generación de solicitud estructurada
\end{itemize}

\subsection{Asignación Inteligente del Prestador}

\subsubsection{Algoritmo de Scoring}

Se calcula un \textbf{score de compatibilidad} para cada prestador disponible:

\begin{equation}
    \text{Score} = w_1 \cdot d + w_2 \cdot s + w_3 \cdot r + w_4 \cdot a + w_5 \cdot c
\end{equation}

Donde:
\begin{itemize}
    \item $d$: Distancia (inverso normalizado)
    \item $s$: Coincidencia de servicio
    \item $r$: Calificación histórica (1-5 estrellas)
    \item $a$: Disponibilidad actual
    \item $c$: Historial de cancelaciones (penalización)
\end{itemize}

Pesos recomendados:
\begin{center}
\begin{tabular}{|l|c|}
    \toprule
    \textbf{Factor} & \textbf{Peso} \\
    \midrule
    Distancia & 0.30 \\
    Servicio & 0.25 \\
    Reputación & 0.25 \\
    Disponibilidad & 0.15 \\
    Cancelaciones & 0.05 \\
    \bottomrule
\end{tabular}
\end{center}

\subsubsection{Proceso de Asignación}

\begin{enumerate}
    \item Calcular score para todos los prestadores candidatos
    \item Ordenar descendentemente
    \item Enviar notificación al DE MAYOR SCORE
    \item Prestador tiene 30 segundos para aceptar/rechazar
    \item Si rechaza, continúa con el siguiente
    \item Si nadie acepta, solicitud entra en cola
\end{enumerate}

\subsection{Seguimiento del Servicio}

\subsubsection{Estados del Servicio}

\begin{center}
\begin{tabular}{|l|p{5cm}|}
    \toprule
    \textbf{Estado} & \textbf{Descripción} \\
    \midrule
    Solicitado & Usuario envió solicitud, esperando asignación \\
    Asignado & Prestador aceptó y confirma llegada \\
    En Camino & Prestador en ruta (GPS activo) \\
    En Ejecución & Prestador en ubicación, ejecutando servicio \\
    Finalizado & Servicio completado, esperando evaluación \\
    Cancelado & Usuario o prestador canceló \\
    \bottomrule
\end{tabular}
\end{center}

\subsubsection{Funcionalidades de Tracking}

\begin{itemize}
    \item \textbf{GPS en tiempo real:} Ubicación del prestador actualizada cada 10 segundos
    \item \textbf{ETA estimado:} Tiempo de llegada calculado dinámicamente
    \item \textbf{Notificaciones push:} Estado, ETA, cambios
    \item \textbf{Historial visible:} Timeline de eventos
    \item \textbf{Comunicación:} Chat o llamada directa
\end{itemize}

\subsection{Sistema de Pagos}

\subsubsection{Características}

\begin{itemize}
    \item El usuario paga \textbf{antes o después} del servicio (configurable)
    \item Plataforma retiene comisión (10-20%)
    \item Prestador recibe el neto
    \item Generación de recibo digital automático
\end{itemize}

\subsubsection{Integraciones Previstas}

\begin{itemize}
    \item \textbf{Nequi:} App bancaria, pagos P2P
    \item \textbf{Daviplata:} Billetera digital
    \item \textbf{Bre-B:} Plataforma de pagos
    \item \textbf{Transferencia bancaria:} Débito de cuenta
    \item \textbf{Pasarela (Wompi/Adyen):} Agregador de métodos
\end{itemize}

\subsubsection{Flujo Simplificado}

\begin{enumerate}
    \item Usuario selecciona método de pago
    \item Autoriza transacción
    \item Plataforma debita monto + comisión
    \item Sistema genera orden de pago
    \item Prestador recibe notificación de pago
    \item Historial de transacción disponible
\end{enumerate}

\subsection{Evaluación y Reputación}

\subsubsection{Componentes}

\begin{itemize}
    \item \textbf{Calificación:} 1 a 5 estrellas (obligatoria)
    \item \textbf{Comentario:} Texto opcional (máx. 500 caracteres)
    \item \textbf{Fotos:} Usuario puede adjuntar evidencia
    \item \textbf{Historial:} Visible públicamente en perfil
\end{itemize}

\subsubsection{Impacto en Futuras Asignaciones}

\begin{itemize}
    \item Calificación $\geq 4.5$ estrellas: Factor multiplicador +1.2
    \item Calificación 3.0-4.5: Factor multiplicador 1.0
    \item Calificación $< 3.0$: Factor multiplicador 0.7 (riesgo)
\end{itemize}

\newpage

% ==================== 7. INTELIGENCIA ARTIFICIAL ====================
\section{Inteligencia Artificial en el Sistema}

\subsection{Tipos de IA Implementada}

\begin{itemize}
    \item \textbf{NLP (Natural Language Processing):} Clasificación de texto
    \item \textbf{Machine Learning supervisado:} Modelos entrenados
    \item \textbf{Scoring dinámico:} Cálculos en tiempo real
    \item \textbf{Aprendizaje iterativo:} Mejora con histórico
\end{itemize}

\subsection{Casos de Uso Concretos}

\begin{enumerate}
    \item \textbf{Clasificación de solicitudes:}
    \begin{itemize}
        \item Texto libre $\rightarrow$ Categoría + subcategoría
        \item Modelo: Red neuronal (BERT, GPT-2 fine-tuned)
        \item Precisión objetivo: $\geq 90\%$
    \end{itemize}
    
    \item \textbf{Detección de urgencia:}
    \begin{itemize}
        \item Palabras clave: ``urgente'', ``emergencia'', ``ahorita''
        \item Reglas + scoring probabilístico
    \end{itemize}
    
    \item \textbf{Optimización de matching:}
    \begin{itemize}
        \item Features: distancia, servicio, reputación, disponibilidad
        \item Algoritmo: Gradient Boosting (XGBoost)
    \end{itemize}
    
    \item \textbf{Aprendizaje del histórico:}
    \begin{itemize}
        \item Aumentar peso de pares usuario-prestador que resultaron bien
        \item Feedback loop mensual
    \end{itemize}
\end{enumerate}

\subsection{Arquitectura del Servicio IA}

\begin{lstlisting}[language=bash]
IA Service (Microservicio Independiente)
├── Python 3.10+
├── FastAPI (servidor HTTP)
├── Hugging Face (transformers preentrenados)
├── Scikit-learn (ML clásico)
├── PostgreSQL (histórico de entrenamientos)
└── Docker (containerización)
\end{lstlisting}

\newpage

% ==================== 8. ARQUITECTURA TECNOLÓGICA ====================
\section{Arquitectura Tecnológica Recomendada}

\subsection{Visión General}

La arquitectura recomendada es una \textbf{Arquitectura de Microservicios con Clean Architecture}, que proporciona:

\begin{itemize}
    \item Escalabilidad horizontal
    \item Separación de responsabilidades
    \item Independencia de deployments
    \item Facilidad de testing
    \item Base sólida académica
\end{itemize}

\subsubsection{Diagrama general (textual)}

\begin{lstlisting}
┌─────────────────────────────────────────┐
│    Apps Móviles (iOS/Android)            │
│    Panel Web Admin                        │
└────────────────┬────────────────────────┘
                 │
        ┌────────┴────────┐
        │   API Gateway   │
        └────────┬────────┘
                 │
    ┌────────────┼────────────┐
    │            │            │
┌──────┐  ┌──────────┐  ┌────────────┐
│ Auth │  │ Users &  │  │ Services & │
│      │  │ Profiles │  │  Matching  │
└──────┘  └──────────┘  └────────────┘
    │            │            │
    └────────────┼────────────┘
                 │
    ┌────────────┼────────────┐
    │            │            │
┌────────┐  ┌────────┐  ┌──────────┐
│ Payments│  │Ratings │  │Tracking  │
└────────┘  └────────┘  └──────────┘
    │            │            │
    └────────────┼────────────┘
                 │
    ┌────────────┴────────────┐
    │  Bases de Datos         │
    │  Cache (Redis)          │
    │  Message Queue (RabbitMQ)│
    └─────────────────────────┘
\end{lstlisting}

\subsection{Frontend}

\subsubsection{Aplicación Móvil}

\begin{itemize}
    \item \textbf{Framework:} Flutter
    \item \textbf{Lenguaje:} Dart
    \item \textbf{Razones:}
    \begin{itemize}
        \item Un solo código base para iOS y Android
        \item Performance excelente
        \item Ideal para mapas y tracking en tiempo real
        \item Comunidad grande y bien soportada
    \end{itemize}
    \item \textbf{Funcionalidades:}
    \begin{itemize}
        \item Login / Registro con Firebase Auth
        \item Solicitud de servicio con geolocalización
        \item Tracking GPS en tiempo real
        \item Integración de pagos
        \item Historial y calificaciones
    \end{itemize}
\end{itemize}

\subsubsection{Panel Administrativo Web}

\begin{itemize}
    \item \textbf{Framework:} React.js
    \item \textbf{Lenguaje:} TypeScript
    \item \textbf{Stack:}
    \begin{itemize}
        \item Vite o Next.js (bundler)
        \item Material-UI o Ant Design (componentes)
        \item Redux o Zustand (state management)
        \item React Query (data fetching)
    \end{itemize}
    \item \textbf{Funcionalidades:}
    \begin{itemize}
        \item Dashboard de métricas
        \item Gestión de usuarios
        \item Verificación de prestadores
        \item Moderación y resolución de conflictos
        \item Reportes analíticos
    \end{itemize}
\end{itemize}

\subsection{Backend}

\subsubsection{Tecnología Base}

\begin{itemize}
    \item \textbf{Lenguaje:} Node.js + TypeScript
    \item \textbf{Framework:} NestJS
    \item \textbf{Razones:}
    \begin{itemize}
        \item Arquitectura modular y escalable
        \item Inspirado en Spring (familiar para devs Java)
        \item Excelente soporte para microservicios
        \item Tipado fuerte con TypeScript
        \item Decentemente documentado
    \end{itemize}
\end{itemize}

\subsubsection{Microservicios Clave}

\paragraph{1. Auth Service}
\begin{itemize}
    \item Login / Logout
    \item JWT tokens + Refresh tokens
    \item OAuth (Google, Facebook)
    \item Roles y permisos
\end{itemize}

\paragraph{2. User Service}
\begin{itemize}
    \item CRUD de usuarios solicitantes
    \item CRUD de prestadores
    \item Perfil y actualización de datos
    \item Gestión de estados
\end{itemize}

\paragraph{3. Service Request Service}
\begin{itemize}
    \item Crear / editar / eliminar solicitudes
    \item Cambio de estados
    \item Historial de solicitudes
    \item Eventos del servicio
\end{itemize}

\paragraph{4. Matching Service}
\begin{itemize}
    \item Algoritmo de asignación
    \item Cálculo de scores
    \item Notificación a prestadores
    \item Aceptación / rechazo
\end{itemize}

\paragraph{5. IA Service (Python separado)}
\begin{itemize}
    \item NLP para clasificación
    \item Detección de urgencia
    \item Integración con modelos ML
\end{itemize}

\paragraph{6. Payment Service}
\begin{itemize}
    \item Procesamiento de pagos
    \item Integración con pasarelas
    \item Cálculo de comisiones
    \item Transferencias a prestadores
\end{itemize}

\paragraph{7. Rating Service}
\begin{itemize}
    \item Calificaciones y comentarios
    \item Cálculo de reputación
    \item Historial de evaluaciones
\end{itemize}

\paragraph{8. Tracking Service (WebSocket)}
\begin{itemize}
    \item GPS en tiempo real
    \item Socket.IO para conexiones bidireccionales
    \item Cálculo de ETA
\end{itemize}

\subsection{Bases de Datos}

\subsubsection{PostgreSQL (Principal)}

\begin{itemize}
    \item Almacenamiento relacional
    \item Tablas: Users, Providers, ServiceRequests, Payments, Ratings, etc.
    \item Transacciones ACID
    \item Backups automáticos
\end{itemize}

\subsubsection{Redis (Cache)}

\begin{itemize}
    \item Cache de sesiones
    \item Rate limiting
    \item Cache de queries frecuentes
    \item Queues de trabajos
\end{itemize}

\subsubsection{MongoDB / Firebase (Eventos)}

\begin{itemize}
    \item Logs de eventos
    \item Tracking de cambios de estado
    \item Historial de interacciones
    \item Escalabilidad de escrituras
\end{itemize}

\subsection{Comunicación en Tiempo Real}

\subsubsection{WebSockets}

\begin{itemize}
    \item \textbf{Librería:} Socket.IO (Node.js)
    \item \textbf{Usos:}
    \begin{itemize}
        \item Ubicación en vivo del prestador
        \item Estado del servicio actualizado
        \item ETA dinámico
        \item Chat entre usuario y prestador
    \end{itemize}
    \item \textbf{Escalado:} Redis Adapter para clustering
\end{itemize}

\newpage

% ==================== 9. INTELIGENCIA ARTIFICIAL (AMPLIADO) ====================
\section{Implementación de Inteligencia Artificial}

\subsection{Tecnologías Específicas}

\begin{itemize}
    \item \textbf{Lenguaje:} Python 3.10+
    \item \textbf{Framework:} FastAPI
    \item \textbf{ML:} Scikit-learn, TensorFlow/PyTorch
    \item \textbf{NLP:} Hugging Face Transformers
    \item \textbf{Modelos preentrenados:} BERT, GPT-2, DistilBERT
\end{itemize}

\subsection{Flujo de Datos IA}

\begin{enumerate}
    \item Usuario escribe solicitud en app móvil: ``Tengo una fuga de agua''
    \item App envía texto a IA Service (POST /classify)
    \item IA Service recibe, procesa y retorna JSON:
    \begin{lstlisting}[language=json]
{
  "categoria": "plomeria",
  "urgencia": 0.85,
  "confianza": 0.92
}
    \end{lstlisting}
    \item Backend actualiza la solicitud con este resultado
    \item Matching Service usará estos datos para asignación
\end{enumerate}

\subsection{Modelos de ML Entrenados Localmente}

Para garantizar:
\begin{itemize}
    \item Control sobre datos sensibles (privacidad)
    \item Independencia de APIs externas
    \item Customización según contexto local
\end{itemize}

Se recomienda fine-tuning de modelos BERT sobre corpus de solicitudes reales.

\newpage

% ==================== 10. SEGURIDAD ====================
\section{Seguridad del Sistema}

\subsection{Autenticación y Autorización}

\subsubsection{Autenticación}

\begin{itemize}
    \item \textbf{JWT (JSON Web Tokens):} Token de acceso (15 min)
    \item \textbf{Refresh Tokens:} Token de renovación (7 días)
    \item \textbf{OAuth 2.0:} Integración con Google y Facebook
    \item \textbf{SMS OTP:} Para operaciones sensibles
\end{itemize}

\subsubsection{Autorización (RBAC)}

\begin{center}
\begin{tabular}{|l|p{8cm}|}
    \toprule
    \textbf{Rol} & \textbf{Permisos} \\
    \midrule
    USER & Crear solicitudes, ver estado, calificar, pagar \\
    PROVIDER & Ver solicitudes, aceptar/rechazar, ejecutar, recibir pago \\
    ADMIN & Acceso total, moderación, reportes \\
    MODERATOR & Resolver disputas, revisar reportes \\
    \bottomrule
\end{tabular}
\end{center}

\subsection{Cifrado y Protección de Datos}

\begin{itemize}
    \item \textbf{HTTPS/TLS:} Todas las comunicaciones encriptadas
    \item \textbf{Hash de contraseñas:} bcrypt con salt
    \item \textbf{Encriptación en reposo:} Datos sensibles en BD
    \item \textbf{AES-256:} Para documentos (cédula, certificados)
\end{itemize}

\subsection{Verificación de Antecedentes}

\subsubsection{Marco Legal (Colombia)}

\begin{quotation}
    ❌ \textbf{NO es legal} acceder directamente a bases de datos de antecedentes penales privadas o estatales sin autorización específica.
\end{quotation}

\subsubsection{Forma Correcta y Legal}

\paragraph{Opción 1: Certificado Judicial (RECOMENDADA para MVP)}

\begin{enumerate}
    \item Prestador autoriza voluntariamente
    \item Sube \textbf{certificado judicial oficial} (emitido por Policía Nacional/Fiscalía)
    \item Plataforma valida:
    \begin{itemize}
        \item Documento existe
        \item Firma digital es válida
        \item Fecha está dentro de rango aceptable (< 2 años)
    \end{itemize}
    \item Marca perfil como \textbf{``Verificado''}
    \item Guarda hash del documento (no el documento en texto)
\end{enumerate}

\paragraph{Opción 2: Integración con Terceros}

Empresas especializadas en background check:
\begin{itemize}
    \item \textbf{Cumple con leyes de protección de datos}
    \item Consultan antecedentes legalmente
    \item Devuelven resultado booleano: ``APTO'' / ``NO APTO''
    \item Ejemplos: Checkify, PermanenterTest
\end{itemize}

\paragraph{Opción 3 (para desarrollo)}

Validación manual + documental:
\begin{itemize}
    \item Admin revisa manualmente documento
    \item Aprueba o rechaza
    \item Decisión registrada en auditoría
\end{itemize}

\subsubsection{Estados de Verificación}

\begin{center}
\begin{tabular}{|l|p{6cm}|}
    \toprule
    \textbf{Estado} & \textbf{Significado} \\
    \midrule
    NO\_VERIFICADO & No ha cargado documentos \\
    EN\_REVISION & Documento cargado, esperando validación \\
    VERIFICADO & Aprobado, puede recibir solicitudes \\
    RECHAZADO & No cumple criterios de seguridad \\
    SUSPENDIDO & Antecedentes negativos o incumplimiento de normas \\
    \bottomrule
\end{tabular}
\end{center}

\subsection{Auditoría y Logging}

\begin{itemize}
    \item Todos los eventos sensibles son registrados
    \item Tabla de auditoría inmutable
    \item Retención: 2 años
    \item Logs de acceso a datos sensibles
\end{itemize}

\newpage

% ==================== 11. DEVOPS E INFRAESTRUCTURA ====================
\section{DevOps e Infraestructura}

\subsection{Containerización}

\subsubsection{Docker}

Cada microservicio tiene su propio Dockerfile:

\begin{lstlisting}[language=bash]
# Ejemplo: Auth Service
FROM node:18-alpine
WORKDIR /app
COPY package*.json ./
RUN npm ci --production
COPY . .
EXPOSE 3001
CMD ["npm", "start"]
\end{lstlisting}

\subsubsection{Docker Compose}

Para desarrollo local (MVP):

\begin{lstlisting}[language=yaml]
version: '3.8'
services:
  auth-service:
    build: ./auth
    ports:
      - "3001:3001"
    
  user-service:
    build: ./users
    ports:
      - "3002:3002"
  
  postgres:
    image: postgres:15
    environment:
      POSTGRES_PASSWORD: secret
    volumes:
      - postgres_data:/var/lib/postgresql/data
  
  redis:
    image: redis:7

volumes:
  postgres_data:
\end{lstlisting}

\subsection{Orquestación (Fase Posterior)}

\subsubsection{Kubernetes}

Para producción con alto tráfico:

\begin{itemize}
    \item Pods para cada microservicio
    \item Ingress Controller para API Gateway
    \item Persistent Volumes para BD
    \item Horizontal Pod Autoscaler
\end{itemize}

\subsection{Cloud Provider}

\subsubsection{Opciones Recomendadas}

\begin{itemize}
    \item \textbf{AWS EC2:} VMs tradicionales, escalables
    \item \textbf{AWS ECS:} Orquestación de contenedores gestionada
    \item \textbf{Google Cloud Run:} Serverless para IA Service
    \item \textbf{Azure App Service:} Para backend
\end{itemize}

\subsection{CI/CD Pipeline}

\subsubsection{GitHub Actions / GitLab CI}

\begin{lstlisting}[language=yaml]
name: Build and Deploy
on:
  push:
    branches: [main, develop]

jobs:
  build:
    runs-on: ubuntu-latest
    steps:
      - uses: actions/checkout@v2
      - name: Run tests
        run: npm test
      - name: Build Docker
        run: docker build -t myapp .
      - name: Push to Registry
        run: docker push myapp:latest
  
  deploy:
    needs: build
    runs-on: ubuntu-latest
    steps:
      - name: Deploy to Cloud
        run: kubectl apply -f deployment.yaml
\end{lstlisting}

\newpage

% ==================== 12. MODELO DE NEGOCIO ====================
\section{Modelo de Negocio}

\subsection{Estructura de Ingresos}

\subsubsection{Comisión por Transacción}

\begin{itemize}
    \item \textbf{Porcentaje:} 10-20\% del monto del servicio
    \item \textbf{Quien paga:} Prestador (descuento neto)
    \item \textbf{Ejemplo:}
    \begin{itemize}
        \item Servicio cuesta: \$100.000
        \item Comisión (15\%): \$15.000
        \item Prestador recibe: \$85.000
    \end{itemize}
\end{itemize}

\subsubsection{Servicios Adicionales (Fase 2)}

\begin{itemize}
    \item \textbf{Suscripción Premium:} Acceso prioritario a solicitudes
    \item \textbf{Featured Profile:} Perfil destacado en búsquedas
    \item \textbf{Publicidad:} Prestadores pueden pagar por visibilidad
\end{itemize}

\subsubsection{Alianzas Comerciales}

\begin{itemize}
    \item Asociaciones con distribuidores de materiales
    \item Comisión por referidos
    \item Seguros para usuarios
\end{itemize}

\subsection{Proyecciones Financieras (Teórico)}

\begin{center}
\begin{tabular}{|l|c|c|c|}
    \toprule
    \textbf{Métrica} & \textbf{Mes 1} & \textbf{Mes 6} & \textbf{Mes 12} \\
    \midrule
    Usuarios activos & 100 & 1.000 & 5.000 \\
    Prestadores & 20 & 200 & 800 \\
    Solicitudes/mes & 150 & 1.500 & 8.000 \\
    Ticket promedio & \$80K & \$80K & \$85K \\
    Ingresos/mes & \$1.8M & \$18M & \$102M \\
    \bottomrule
\end{tabular}
\end{center}

\newpage

% ==================== 13. PLAN DE IMPLEMENTACIÓN ====================
\section{Plan de Implementación}

\subsection{Fases del Proyecto}

\subsubsection{Fase 0: Evaluación y Setup (Semanas 1-2)}

\begin{itemize}
    \item Definición de requisitos técnicos
    \item Configuración de repositorios
    \item Setup de ambientes (dev, staging, prod)
    \item Selección de herramientas
\end{itemize}

\subsubsection{Fase 1: MVP Base (Semanas 3-12)}

\begin{itemize}
    \item \textbf{Auth Service:} Login, JWT
    \item \textbf{User Service:} CRUD usuarios y prestadores
    \item \textbf{Service Request Service:} Crear y listar solicitudes
    \item \textbf{Frontend básico:} Login, crear solicitud, listar
    \item Base de datos PostgreSQL
    \item Tests unitarios
\end{itemize}

\subsubsection{Fase 2: Matching e IA (Semanas 13-20)}

\begin{itemize}
    \item Matching Service con algoritmo básico
    \item IA Service (NLP) - MVP
    \item Notificaciones push
    \item Tracking básico (no GPS)
\end{itemize}

\subsubsection{Fase 3: Pagos y Tracking (Semanas 21-28)}

\begin{itemize}
    \item Payment Service con pasarela
    \item GPS tracking en tiempo real
    \item WebSocket para actualizaciones
    \item Rating Service
\end{itemize}

\subsubsection{Fase 4: Panel Admin y Optimizaciones (Semanas 29-36)}

\begin{itemize}
    \item Panel administrativo web
    \item Dashboard de métricas
    \item Herramientas de moderación
    \item Optimización de performance
    \item Tests E2E
\end{itemize}

\subsubsection{Fase 5: Producción y Escalado (Semanas 37+)}

\begin{itemize}
    \item Deployment a cloud
    \item CI/CD pipeline
    \item Monitoreo y alertas
    \item Optimización y escalado
\end{itemize}

\subsection{Equipo Recomendado}

\begin{center}
\begin{tabular}{|l|c|p{5cm}|}
    \toprule
    \textbf{Rol} & \textbf{Cantidad} & \textbf{Responsabilidades} \\
    \midrule
    Backend Developer & 2 & NestJS, APIs, BD \\
    Frontend Developer & 2 & Flutter, React \\
    DevOps Engineer & 1 & Infraestructura, CI/CD \\
    ML Engineer & 1 & NLP, IA \\
    QA Engineer & 1 & Testing, calidad \\
    Product Manager & 1 & Visión, requisitos \\
    Tech Lead & 1 & Arquitectura, decisiones \\
    \bottomrule
\end{tabular}
\end{center}

\newpage

% ==================== 14. IMPACTO ESPERADO ====================
\section{Impacto Esperado}

\subsection{Impacto Social}

\begin{itemize}
    \item \textbf{Formalización laboral:} Trabajadores independientes con visibilidad formal
    \item \textbf{Confianza:} Usuarios confían en prestadores verificados
    \item \textbf{Reducción de fraude:} Sistema de reputación desincentiva estafas
    \item \textbf{Inclusión financiera:} Acceso a pagos digitales
\end{itemize}

\subsection{Impacto Económico}

\begin{itemize}
    \item \textbf{Determinación de ingresos:} Prestadores tienen fuente estable
    \item \textbf{Escalabilidad:} Usuarios pueden acceder a servicios sin buscar informalmente
    \item \textbf{Creación de empleo:} Demanda de desarrolladores, soporte, operaciones
    \item \textbf{Transacciones formales:} Todo queda registrado y tributario
\end{itemize}

\subsection{Impacto Tecnológico}

\begin{itemize}
    \item \textbf{IA aplicada:} Demostración real de NLP en contexto local
    \item \textbf{Arquitectura moderna:} Microservicios, clean architecture
    \item \textbf{Stack contemporáneo:} Node.js, Flutter, PostgreSQL, Python
    \item \textbf{Cloud-native:} Preparada para escalar globalmente
\end{itemize}

\subsection{Impacto Académico}

\begin{itemize}
    \item Proyecto integradora de conocimientos de ingeniería de software
    \item Aplicación de patrones de arquitectura
    \item Demostración de capacidad de ejecución empresarial
    \item Base sólida para publicaciones o spin-off académico
\end{itemize}

\newpage

% ==================== 15. RIESGOS Y MITIGACIÓN ====================
\section{Riesgos y Plan de Mitigación}

\subsection{Riesgos Técnicos}

\begin{center}
\begin{tabular}{|p{3cm}|p{3cm}|p{4cm}|}
    \toprule
    \textbf{Riesgo} & \textbf{Probabilidad} & \textbf{Mitigación} \\
    \midrule
    Complejidad de IA mayor a la esperada & MEDIA & Usar modelos preentrenados, no entrenar from scratch \\
    Performance de matching en tiempo real & MEDIA & Caching, índices BD, algoritmo optimizado \\
    Escalado de BD & BAJA & Sharding temprano, réplicas \\
    \bottomrule
\end{tabular}
\end{center}

\subsection{Riesgos de Negocio}

\begin{center}
\begin{tabular}{|p{3cm}|p{3cm}|p{4cm}|}
    \toprule
    \textbf{Riesgo} & \textbf{Probabilidad} & \textbf{Mitigación} \\
    \midrule
    Baja adopción inicial & ALTA & Marketing local, incentivos iniciales \\
    Competencia existente & MEDIA & Diferenciación en IA y UX \\
    Regulación de plataformas & MEDIA & Cumplimiento legal de datos, seguridad \\
    \bottomrule
\end{tabular}
\end{center}

\newpage

% ==================== 16. CONCLUSIONES ====================
\section{Conclusiones}

\subsection{Síntesis}

CameYa/CameYo representa una \textbf{oportunidad integral} de innovación tecnológica que combina:

\begin{enumerate}
    \item Una \textbf{problemática real} (mercado lineal y desorganizado)
    \item Una \textbf{solución tecnológica sólida} (microservicios, IA, pagos)
    \item Un \textbf{modelo de negocio viable} (comisión, escalable)
    \item Una \textbf{arquitectura robusta y académica} (Clean Arch, SOLID)
\end{enumerate}

\subsection{Viabilidad}

\begin{itemize}
    \item ✅ Técnicamente viable: Tecnologías disponibles y maduras
    \item ✅ Económicamente viable: Modelo de negocio comprobado en otras plataformas
    \item ✅ Académicamente sólido: Cumple estándares de proyectos de grado
    \item ✅ Escalable: Arquitectura preparada para crecer
\end{itemize}

\subsection{Recomendaciones Finales}

\begin{enumerate}
    \item \textbf{Comenzar con MVP enfocado}: Solicitud + Matching + Pagos básicos
    \item \textbf{Validar con usuarios reales}: Pilotos en 1-2 ciudades
    \item \textbf{Fine-tuning de IA en histórico real}: No confiar solo en datos públicos
    \item \textbf{Énfasis en seguridad}: Verificación de antecedentes es crítica
    \item \textbf{Escalado gradual}: No intentar ser Uber en mes 1
\end{enumerate}

\subsection{Próximos Pasos}

A partir de este documento, se recomienda:

\begin{enumerate}
    \item Diseño detallado de base de datos
    \item Especificación de APIs (OpenAPI/Swagger)
    \item Diagramas UML de arquitectura
    \item Prototipo de UI/UX
    \item Plan de seguridad detallado
    \item Roadmap técnico en Jira/Asana
\end{enumerate}

\newpage

% ==================== REFERENCIAS Y APÉNDICES ====================
\section{Referencias}

\begin{enumerate}
    \item Fowler, M. (2014). \textit{Microservices}. martinfowler.com
    \item Evans, E. (2003). \textit{Domain-Driven Design}. Addison-Wesley
    \item Martin, R. C. (2017). \textit{Clean Architecture}. Prentice Hall
    \item Goodfellow, I., Bengio, Y., \& Courville, A. (2016). \textit{Deep Learning}. MIT Press
    \item Devlin, J., et al. (2018). BERT: Pre-training of Deep Bidirectional Transformers
\end{enumerate}

\appendix

\section{Tabla Comparativa de Tecnologías}

\begin{center}
\begin{tabular}{|l|l|l|l|}
    \toprule
    \textbf{Aspecto} & \textbf{Opción 1} & \textbf{Opción 2} & \textbf{Seleccionada} \\
    \midrule
    Frontend & React Native & \textbf{Flutter} & Flutter \\
    Backend & Express & \textbf{NestJS} & NestJS \\
    BD Relacional & MySQL & \textbf{PostgreSQL} & PostgreSQL \\
    Cache & Memcached & \textbf{Redis} & Redis \\
    Mensajería & Kafka & \textbf{RabbitMQ} & RabbitMQ \\
    IA Framework & TensorFlow & \textbf{Hugging Face} & Hugging Face \\
    Cloud & AWS & GCP & \textbf{Flexible} \\
    \bottomrule
\end{tabular}
\end{center}

\section{Glosario de Términos}

\begin{itemize}
    \item \textbf{API Gateway:} Punto de entrada unificado para todas las APIs
    \item \textbf{Microservicio:} Componente independiente que realiza una función específica
    \item \textbf{JWT:} JSON Web Token para autenticación
    \item \textbf{NLP:} Natural Language Processing, procesamiento de lenguaje natural
    \item \textbf{OTP:} One-Time Password, contraseña única
    \item \textbf{RBAC:} Role-Based Access Control, control de acceso basado en roles
    \item \textbf{Matching:} Proceso de asignación de prestador a solicitud
    \item \textbf{Scoring:} Cálculo de puntuación de compatibilidad
\end{itemize}

\end{document}
